\documentclass[a4paper,12pt]{article}

\usepackage[latin1]{inputenc}
\usepackage{hyperref}
\usepackage{tabularx}

\renewcommand{\arraystretch}{1.5}

\title{LibNGS}
\author{Sylvain FOR\^ET}

\begin{document}

\maketitle

\newpage

\tableofcontents

\newpage

%%%%%%%%%%%%%%%%%%%%%%%
%%%%%%%%%%%%%%%%%%%%%%%
%%%%%%%%%%%%%%%%%%%%%%%
%%%                 %%%
%%% Getting started %%%
%%%                 %%%
%%%%%%%%%%%%%%%%%%%%%%%
%%%%%%%%%%%%%%%%%%%%%%%
%%%%%%%%%%%%%%%%%%%%%%%

\section{Getting started}

\subsection{Pre-requisite}

\paragraph{flex} In order to compile libngs and its tools you will need flex
(the Fast Lexical Analyser).  In most Linux distributions, this comes in the
`flex' package.  For more information, see \url{http://flex.sourceforge.net/}.

\paragraph{libglib} You will also need libglib and its header files in a
version greater or equal to 2.16.  On a Debian system, this can be found in the
libglib2.0-0 and libglib2.0-0-dev packages.  Other distributions probably use
similar names.  For more information, see \url{http://www.gtk.org/}.

\subsection{Installation}

\paragraph{}
In brief, for a system-wide installation:

\begin{verbatim}
./autogen.sh
./configure
make
sudo make install
\end{verbatim}

Note: you only need to run \texttt{./autogen.sh} if the \texttt{configure} file
does not exist.

\paragraph{}
For a user-specific installation:

\begin{verbatim}
./autogen.sh
./configure --prefix=/somewhere/where/you/have/write/access
make install
\end{verbatim}


%%%%%%%%%%%%%%%%%%
%%%%%%%%%%%%%%%%%%
%%%%%%%%%%%%%%%%%%
%%%            %%%
%%% Philosophy %%%
%%%            %%%
%%%%%%%%%%%%%%%%%%
%%%%%%%%%%%%%%%%%%
%%%%%%%%%%%%%%%%%%

\section{Philosophy}

%%%%%%%%%%%%
%%%%%%%%%%%%
%%        %%
%% Design %%
%%        %%
%%%%%%%%%%%%
%%%%%%%%%%%%

\subsection{Design}

\paragraph{}
The central component of this software is a library, located in
\texttt{src/libns}.
This library includes the following components:
\begin{itemize}
    \item Data structures
    \item IO routines
    \item Some algorithms (mainly kmer hashing for now)
\end{itemize}
The bulk of the library is written in C.
The parsers are automatically generated with LEX.

\paragraph{}
Various binary programs are implemented using this libraries.
These can be found in \texttt{src/bin}.

\paragraph{}
Finally, a few shell, python and R scripts wrapping these binary utilities or
using their outputs are also provided.
These scripts are located in \texttt{src/scripts}.

\paragraph{}
Examples of pipelines using the various scripts and binaries can be found in
\texttt{src/pipelines}.
See also the description of pipelines in section \ref{pipelines}.

%%%%%%%%%%%
%%%%%%%%%%%
%%       %%
%% Usage %%
%%       %%
%%%%%%%%%%%
%%%%%%%%%%%

\subsection{Usage}

\paragraph{}
All the utilities are design to read from the standard input and to write the
standard output.
This allows you to chain various commands together using pipes.
Example 1:
\begin{verbatim}
bzcat infile.bz2 |
    fastq_sample -n 50 - |
    fastq_trim -q 20 - |
    gzip -c - > outfile.gz
\end{verbatim}
This does the following:
\begin{enumerate}
    \item \texttt{bzcat infile.bz2}: decompresses a bzip2 compressed file to
        the standard output.
    \item \texttt{fastq\_sample -n 50}: samples 50 sequence from the standard
        input.
    \item \texttt{fastq\_trim -q 20}: trims the bases of quality less than 20.
    \item \texttt{gzip -c -}: compresses the stream to gzip format.
\end{enumerate}

\paragraph{}
When using programs that read or write more than one file at a time, one should take
advantage of the named pipes (FIFOs) available in Unix systems.
Example 2:
\begin{verbatim}
mkfifo pipe1 pipe2
zcat infile1.gz | fastq_trim -q 20 -o pipe1 - &
zcat infile1.gz | fastq_trim -q 20 -o pipe2 - &
fastq_pairs pipe1 pipe2 | gzip -c - > outfile.gz &
wait
rm pipe1 pipe2
\end{verbatim}
This does the following:
\begin{enumerate}
    \item \texttt{mkfifo pipe1 pipe2}: creates the named pipes.
    \item \texttt{zcat infile1.gz}: decompresses a gzip compressed file to the
        standard output.
    \item \texttt{fastq\_trim -q 20 -o pipe1}: quality trims the sequences, and
        sends the output to the named pipe `\texttt{pipe1}'
    \item The same thing is done for the other file (e.g. for paired end reads
        present in two separate files).  Both tasks are sent to the background
        with the `\&' symbol at the end of each line, and are executed
        simultaneously.
    \item \texttt{fastq\_pairs pipe1 pipe2}: both named pipes are read and
        combined into a file where pairs of reads are interleaved.
    \item \texttt{wait}: waits  for the commands in the background to finish.
    \item \texttt{rm pipe1 pipe2}: cleans up the temporary named pipes.
\end{enumerate}

Additional examples can be found in the \texttt{pipelines} directory (see
section \ref{pipelines}).

%%%%%%%%%%%%%%%%
%%%%%%%%%%%%%%%%
%%%%%%%%%%%%%%%%
%%%          %%%
%%% Programs %%%
%%%          %%%
%%%%%%%%%%%%%%%%
%%%%%%%%%%%%%%%%
%%%%%%%%%%%%%%%%

\section{Programs}

Description of all the programs.

%%%%%%%%%%%%%%%%%%%%%%%%
%%%%%%%%%%%%%%%%%%%%%%%%
%%                    %%
%% Quality assessment %%
%%                    %%
%%%%%%%%%%%%%%%%%%%%%%%%
%%%%%%%%%%%%%%%%%%%%%%%%

\subsection{Quality assessment}

\subsubsection{summary}


\begin{tabularx}{\textwidth}{|X|X|}
    \hline
    \textbf{Name}                   & \textbf{Description} \\
    \hline
    \hline
    fastq\_base\_qual\_summary      & Global distribution of base qualities \\
    fastq\_letter\_pos              & Distribution of letters along the length of sequences \\
    fastq\_letter\_qual             & Global distribution of base qualities, by letter \\
    fastq\_qual\_length\_summary    & Distribution of qualities along the length of sequences \\
    fastq\_read\_qual\_summary      & Distribution of average sequence qualities \\
    \hline
\end{tabularx}

\subsubsection{fastq\_base\_qual\_summary}

Global distribution of base qualities.

\subsubsection{fastq\_letter\_pos}

Distribution of letters along the length of sequences.

\subsubsection{fastq\_letter\_qual}

Global distribution of base qualities, by letter.

\subsubsection{fastq\_qual\_length\_summary}

Distribution of qualities along the length of sequences.

\subsubsection{fastq\_read\_qual\_summary}

Distribution of qualities along the length of sequences.

%%%%%%%%%%%%%%%%%%%%%%%
%%%%%%%%%%%%%%%%%%%%%%%
%%                   %%
%% Sequence handling %%
%%                   %%
%%%%%%%%%%%%%%%%%%%%%%%
%%%%%%%%%%%%%%%%%%%%%%%

\subsection{Sequence handling}

Distribution of average sequence qualities.

\subsubsection{summary}

\begin{tabularx}{\textwidth}{|X|X|}
    \hline
    \textbf{Name}                   & \textbf{Description} \\
    \hline
    \hline
    fastq2fasta                     & Converts a fastq file to a fasta file \\
    fastq2fastq                     & Conversion between different version of fastq format \\
    fastq\_fetch                    & Extracts sequences from a fastq file by name \\
    fastq\_interleave               & Deprecated, use fastq\_pairs instead \\
    fastq\_pairs                    & Interleaves two mate pair / paired end files or the other way around \\
    fastq\_revcomp                  & Reverse complements sequences \\
    fastq\_sample                   & One pass sampling \\
    fastq\_split                    & Splits sequences in two (deals with a old odd version of paired end fastq) \\
    fastq\_trim\_adaptors           & Removes adaptors using approximate word matching \\
    fastq\_trim                     & Quality trimming \\
    \hline
\end{tabularx}

\subsubsection{fastq2fasta}

Converts a fastq file to a fasta file.

\subsubsection{fastq2fastq}

Conversion between different version of fastq format.

\subsubsection{fastq\_fetch}

Extracts sequences from a fastq file by name.

\subsubsection{fastq\_interleave}

Deprecated, use fastq\_pairs instead.

\subsubsection{fastq\_pairs}

Interleaves two mate pair / paired end files or the other way around.

\subsubsection{fastq\_revcomp}

Reverse complements sequences.

\subsubsection{fastq\_sample}

One pass sampling.

\subsubsection{fastq\_split}

Splits sequences in two (deals with a old odd version of paired end fastq).

\subsubsection{fastq\_trim\_adaptors}

Removes adaptors using approximate word matching.

\subsubsection{fastq\_trim}

Quality trimming.

%%%%%%%%%%%%%%%%%%%%%
%%%%%%%%%%%%%%%%%%%%%
%%                 %%
%% K-mer utilities %%
%%                 %%
%%%%%%%%%%%%%%%%%%%%%
%%%%%%%%%%%%%%%%%%%%%

\subsection{K-mer utilities}

\subsubsection{summary}

\begin{tabularx}{\textwidth}{|X|X|}
    \hline
    \textbf{Name}                   & \textbf{Description} \\
    \hline
    \hline
    kmers\_count                    & Counts kmer occurrences in sequences \\
    kmers\_count\_tool              & Manipulates count files (binary or text) \\
    kmers\_remove\_clonal           & Removes identical sequences from Illumina libraries \\
    \hline
\end{tabularx}

\subsubsection{kmers\_count}

Counts kmer occurrences in sequences.

\subsubsection{kmers\_count\_tool}

Manipulates count files (binary or text).

\subsubsection{kmers\_remove\_clonal}

Removes identical sequences from Illumina libraries.
Especially useful for large insert Illumina mate pair libraries in the case of
de-novo assembly.

%%%%%%%%%%%%%%%%%
%%%%%%%%%%%%%%%%%
%%             %%
%% Methylation %%
%%             %%
%%%%%%%%%%%%%%%%%
%%%%%%%%%%%%%%%%%

\subsection{Methylation}

Most of this should be deprecated in favour of more standard formats such as
SAM (for alignments) and BED (for methylation counts).
The core algorithms are still very important, especially regarding Neighbour
Quality Standards and other quality related checks.

\subsubsection{summary}

\begin{tabularx}{\textwidth}{|X|X|}
    \hline
    \textbf{Name}                   & \textbf{Description} \\
    \hline
    \hline
    bsq2gff                         & Should be deprecated, use SAM format instead \\
    bsq\_coverage\_distribution     & Should be deprecated, use SAM format instead \\
    bsq\_empty\_regions\_content    & Should be deprecated, use SAM format instead \\
    bsq\_methylation\_counts        & Should be deprecated, use SAM format instead \\
    bsq\_summary                    & Should be deprecated, use SAM format instead \\
    cg\_fetch                       & Bed format might be preferable (standard) even if it is less compact \\
    cg\_merge                       & Bed format might be preferable (standard) even if it is less compact \\
    cg\_meth\_count                 & Bed format might be preferable (standard) even if it is less compact \\
    cg\_meth\_dist                  & Bed format might be preferable (standard) even if it is less compact \\
    \hline
\end{tabularx}

\subsubsection{bsq2gff}

Should be deprecated, use SAM format instead.

\subsubsection{bsq\_coverage\_distribution}

Should be deprecated, use SAM format instead.

\subsubsection{bsq\_empty\_regions\_content}

Should be deprecated, use SAM format instead.

\subsubsection{bsq\_methylation\_counts}

Should be deprecated, use SAM format instead.

\subsubsection{bsq\_summary}

Should be deprecated, use SAM format instead.

\subsubsection{cg\_fetch}

Bed format might be preferable (standard) even if it is less compact.

\subsubsection{cg\_merge}

Bed format might be preferable (standard) even if it is less compact.

\subsubsection{cg\_meth\_count}

Bed format might be preferable (standard) even if it is less compact.

\subsubsection{cg\_meth\_dist}

Bed format might be preferable (standard) even if it is less compact.

%%%%%%%%%%%%%%%%%
%%%%%%%%%%%%%%%%%
%%%%%%%%%%%%%%%%%
%%%           %%%
%%% Pipelines %%%
%%%           %%%
%%%%%%%%%%%%%%%%%
%%%%%%%%%%%%%%%%%
%%%%%%%%%%%%%%%%%

\section{Pipelines}

\label{pipelines}

Pipelines are examples of shell scripts tying various utilities together.
They are not necessarily designed to be used directly, but rather should be
used as templates that can be tuned to one's particular needs.

%%%%%%%%%%%%%%%%%%%%%%%%%%%
%%%%%%%%%%%%%%%%%%%%%%%%%%%
%%                       %%
%% quality_assessment.sh %%
%%                       %%
%%%%%%%%%%%%%%%%%%%%%%%%%%%
%%%%%%%%%%%%%%%%%%%%%%%%%%%

\subsection{quality\_assessment.sh}

\paragraph{}
This script goes sequentially through a number of gzip compressed fastq files,
and for each file, runs various quality assessment utilities in parallel.
A report in pdf format is generated.

\paragraph{}
Possible customisations include the addition or removal of quality assessment
utilities, and running all the files in parallel for systems with enough CPUs.

%%%%%%%%%%%%%%%%%
%%%%%%%%%%%%%%%%%
%%             %%
%% trimming.sh %%
%%             %%
%%%%%%%%%%%%%%%%%
%%%%%%%%%%%%%%%%%

\subsection{trimming.sh}

This script defines a few simple functions that do the typical quality
filtering on single end, paired end and mate pair libraries.
The various quality and length thresholds should be tunned to particular
datasets.

%%%%%%%%%%%%%%%
%%%%%%%%%%%%%%%
%%%%%%%%%%%%%%%
%%%         %%%
%%% Library %%%
%%%         %%%
%%%%%%%%%%%%%%%
%%%%%%%%%%%%%%%
%%%%%%%%%%%%%%%

\section{Library}

Documentation of the functions and structures of the C library.

%%%%%%%%%%%%%%%
%%%%%%%%%%%%%%%
%%%%%%%%%%%%%%%
%%%         %%%
%%% Licence %%%
%%%         %%%
%%%%%%%%%%%%%%%
%%%%%%%%%%%%%%%
%%%%%%%%%%%%%%%

\section{Licence}

This software is available under the terms of the GPL version 3 or later.
A copy of this licence is distributed with the sources.

\end{document}

% vim:ft=tex:spell:ts=4:sw=4:sts=4:expandtab:
%
